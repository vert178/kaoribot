\documentclass[11pt]{article}
\usepackage{geometry}
\geometry
{ a4paper,
total={170mm,240mm},
left=20mm,
top=20mm,}
\setlength{\parindent}{0pt}

\usepackage[english]{babel}
\usepackage{mathtools}
\usepackage{mathdots}
\usepackage{amsthm}
\usepackage{amsmath}
\usepackage{amssymb}
\usepackage{framed}
\usepackage[T1]{fontenc}
\usepackage[super,sort&compress]{natbib}
\usepackage{tikz}
\usepackage{xcolor}
\usepackage{xifthen}
\usepackage{xstring}

\usepackage{pgfplots}
\pgfplotsset{compat=1.17}
\usepgfplotslibrary{fillbetween}

\DeclarePairedDelimiter{\norm}{\lVert}{\rVert}
\DeclarePairedDelimiter{\abs}{\lvert}{\rvert}
\DeclarePairedDelimiter{\floor}{\lfloor}{\rfloor}
\DeclarePairedDelimiter{\ceil}{\lceil}{\rceil}

\newcommand{\defeq}{\vcentcolon=}
\newcommand{\eqdef}{=\vcentcolon}
\newcommand{\limit}[2]{\mathop{\mathrm{lim}}_{{#1} \to {#2}}}
\newcommand{\limitn}{\mathop{\mathrm{lim}}_{n \to \infty}}
\newcommand{\limitsup}{\mathop{\mathrm{lim sup}}_{n \to \infty}}
\newcommand{\limitinf}{\mathop{\mathrm{lim inf}}_{n \to \infty}}
\newcommand{\lproof}{\(\left(\Leftarrow\right)\)}
\newcommand{\rproof}{\(\left(\Rightarrow\right)\)}
\newcommand{\darrow}{\(\Longleftrightarrow\)}
\newcommand{\md}{\: \mathrm{d}}
\newcommand{\mbn}{\mathbb{N}}
\newcommand{\mbr}{\mathbb{R}}
\newcommand{\mbz}{\mathbb{Z}}
\newcommand{\mbq}{\mathbb{Q}}
\newcommand{\mbc}{\mathbb{C}}
\newcommand{\mbf}{\mathbb{F}}
\newcommand{\del}{\partial}
\newcommand{\ddx}[1]{\frac{\mathrm{d}}{\mathrm{d} #1}}
\newcommand{\ds}{\displaystyle}


\setcitestyle{comma,numbers,super,open={},close={}} 
\makeatletter
\renewcommand\@biblabel[1]{#1.}
\makeatother

\renewcommand{\labelenumi}{\alph{enumi})}

%Define headers and stuff
\usepackage{fancyhdr}
\fancypagestyle{firstpage} {
   \renewcommand{\headrulewidth}{0pt}
   \fancyhf{} 
   \fancyfoot[C]{Page \thepage} 
}

\renewcommand{\headrulewidth}{.15mm}  % header line width

\pagestyle{fancy}
\fancyhf{}
\headheight = 40pt
\rhead{\bf{Version 2.1}}
\lhead{\textnormal{\bf{Kaori bot}}}
\cfoot{Page \thepage}

\renewcommand{\v}{\vspace{15pt}}


\begin{document}
\thispagestyle{firstpage}

\begin{center}
\begin{large}
{\textnormal{\bf 
Vert178
}\hspace{2.5pt}$\bullet$\hspace{7pt}\textnormal{\bf 
Kaori bot
}\hspace{2.5pt}$\bullet$\hspace{7pt}\textnormal{\bf 
Documentation
}}\\
\smallskip
\bf{Version 2.1
}\\
\smallskip

\hrulefill
\end{large}
\end{center}

This documentation is a detailed description of the functions implemented in the open-sourced Kaori bot project in the MitK server (discord.gg/5fZanrCdkG). It is intended for the server's committees and members to gain a better understanding of the project.

\tableofcontents



\newpage

\section{Commands}
A command must have a \emph{name} and an \emph{execute} function. The former is the command initiator, while the latter is the function to be executed shall the command passes several tests. There are also several other properties.

\subsection{Name}
\textbf{Type: string} \hspace{5pt} The initiator of the command. With the correct initiator and a valid initiator that matches the name of some command, the command will trigger.

\subsection{Description, Example}
\textbf{Type: string} \hspace{5pt} The description and example of the command, which will be displayed in the help function.

\subsection{Alias}
\textbf{Type: string[]} \hspace{5pt} Other initiators of the command. Equally valid with the name.

\subsection{User Restricted}
\textbf{Type: boolean} \hspace{5pt} Whether the command is restricted to several specific users. Usually they are debug functions reserved only for developers.

\subsection{Role Restricted}
\textbf{Type: boolean} \hspace{5pt} Whether the command is restricted to users with several specific roles. Usually they are moderation functions.

\subsection{Hidden}
\textbf{Type: boolean} \hspace{5pt} Whether the command is not displayed when "Kaori, help" is called.


\subsection{Execute}
\textbf{Type: function(message, args)} \hspace{5pt}
The body of the command to be called and executed if the command is initiated.

\medskip

The strings of text that follows a command are called arguments (args). They are separated by spaces. 

\newpage

\section{Commands}
This section shows all the commands and their example usages.

\subsection{Ping}
\textbf{Description: } \hspace{5pt} Attempts to ping Kaori, just to check if she has arrived\\
\textbf{Alias: } \hspace{5pt} test, wake \\
\textbf{Example: } \hspace{5pt} kaori, ping\\
\textbf{User Restricted: } \hspace{5pt} False\\
\textbf{Role Restricted: } \hspace{5pt} False\\
\textbf{Hidden: } \hspace{5pt} False\\
\textbf{Args: } \hspace{5pt} (debug)
\\ \\
Kaori will display the amount of latency, calculated by subtracting the time on the computer that Kaori is running on, and the snowflake attached to the message that she has received. Note that if the time on my computer lags behind, the time might be negative. \\ \\
User: Kaori, ping
\\
Kaori: Hiya I am 175ms late. Not too bad :)
\\ \\
User: Kaori, ping debug
\\ 
Kaori: Debug: 
\\
Time now is : 1629904466251
\\
Message Created at : 1629904466084 
\\
Took : 167

\subsection{Reload}
\textbf{Description: } \hspace{5pt} Why do you even care \\
\textbf{Alias: } \hspace{5pt} r, load, refresh \\
\textbf{Example: } \hspace{5pt} kaori, reload search \\
\textbf{User Restricted: } \hspace{5pt} False\\
\textbf{Role Restricted: } \hspace{5pt} True\\
\textbf{Hidden: } \hspace{5pt} True\\
\textbf{Args: } \hspace{5pt} The function you want to reload.

A function to delete the cache and reload a function. Useful while debugging only. If, for some reason, Kaori encountered an error while trying to reload a function, she will cry or be mildly infuriated about it.
\\ \\
User: Kaori, reload search
\\
Kaori: (successful reload) search? ok sure
\\ \\
User: Kaori, reload notvalidfunction
\\ 
Kaori: No thats not how it works.

\subsection{Getinfo}
\textbf{Description: } \hspace{5pt} Why do you even care \\
\textbf{Alias: } \hspace{5pt} \\
\textbf{Example: } \hspace{5pt} kaori, getinfo allegro \\
\textbf{User Restricted: } \hspace{5pt} False\\
\textbf{Role Restricted: } \hspace{5pt} True\\
\textbf{Hidden: } \hspace{5pt} True\\
\textbf{Args: } \hspace{5pt} A music glossary search term.

Extracts the vote info of a term. Useful for debug only.
\\ \\
User: Kaori, getinfo allegro
\\ 
Kaori: Info for allegro: 1, 0, 1, 0

\subsection{Tell}
\textbf{Description: } \hspace{5pt} Looks up info for bot-faq function. \\
\textbf{Alias: } \hspace{5pt} answer, t \\
\textbf{Example: } \hspace{5pt} kaori, tell allegro \\
\textbf{User Restricted: } \hspace{5pt} False\\
\textbf{Role Restricted: } \hspace{5pt} False\\
\textbf{Hidden: } \hspace{5pt} False\\
\textbf{Args: } \hspace{5pt} A music glossary search term.

Kaori will attempt to look up the term provided and see if there is a matching entry done by the community. If not, she will attempt a search and prompts the user to manually click the term they are looking for.
\\
After that, she will read the data from the database, and presents the info (Title, Description, Votes etc.) in an embed. The user is able to upvote or downvote using buttons.

\subsection{Help}
\textbf{Description: } \hspace{5pt} Shows all the non-hidden commands and their respective usages \\
\textbf{Alias: } \hspace{5pt} commands \\
\textbf{Example: } \hspace{5pt} kaori, help \\
\textbf{User Restricted: } \hspace{5pt} False\\
\textbf{Role Restricted: } \hspace{5pt} False\\
\textbf{Hidden: } \hspace{5pt} False\\
\textbf{Args: } \hspace{5pt} (command)

Pretty self-explainatory.
\\ \\
User: Kaori, getinfo allegro
\\ 
Kaori: (in chat) Kaori slid into those dms, smooth like butter.
\\
Kaori: (in dms) Use the prefix "Kaori, " before a command. 
\\ \\
 Alternatively, you can use "k " if you'd prefer a shorter prefix \\
======================================================== \\

 Anyway here's a list of stuff that I can do: ping, tell, help, name, search, source
 \\
 Ask again with Kaori, help [command name] if you want more info
 
\subsection{Name}
\textbf{Description: } \hspace{5pt} She will give you a random key, or a few random notes or whatever. Just a funny little command with basically no use. \\
\textbf{Alias: } \hspace{5pt} choose \\
\textbf{Example: } \hspace{5pt} kaori, name 3 notes \\
\textbf{User Restricted: } \hspace{5pt} False\\
\textbf{Role Restricted: } \hspace{5pt} False\\
\textbf{Hidden: } \hspace{5pt} False\\
\textbf{Args: } \hspace{5pt} (number)(coin/note/key)

Gives you back a random key, or a random note, or even a random coin flip.
\\ \\
User: Kaori, name a key
\\ 
Kaori: Ab major
\\ \\
User: Kaori, name 3 notes
\\ 
Kaori: Ab, E, D


\subsection{Search}
\textbf{Description: } \hspace{5pt} Looks up the info of a piece. \\
\textbf{Alias: } \hspace{5pt} find, s \\
\textbf{Example: } \hspace{5pt} kaori, search chopin etude waterfall \\
\textbf{User Restricted: } \hspace{5pt} False\\
\textbf{Role Restricted: } \hspace{5pt} False\\
\textbf{Hidden: } \hspace{5pt} False\\
\textbf{Args: } \hspace{5pt} (name of piece)

Henle system, but better. Now (almost) complete with autofill functions and user-suggested pieces.
\\ \\
User: Kaori, search chopin etude waterfall
\\ 
Kaori: Sure! Found 5 results. Please react to the appropriate number for the result that you wanted.
 1: Chopin, F. - Etude Op 10 Nr 1 "Waterfall"
 2: Chopin, F. - Etude Op 10 Nr 10
 3: Chopin, F. - Etude Op 10 Nr 11
 4: Chopin, F. - Etude Op 10 Nr 12 "Revolutionary"
 5: Chopin, F. - Etude Op 10 Nr 2 "Chromatique"
 \\
User chooses a piece, and then Kaori presents the embed with the info. Happy ending :)

\subsection{Source}
\textbf{Description: } \hspace{5pt} Shows the source code or the data. \\
\textbf{Alias: } \hspace{5pt}  \\
\textbf{Example: } \hspace{5pt} kaori, source code \\
\textbf{User Restricted: } \hspace{5pt} False\\
\textbf{Role Restricted: } \hspace{5pt} False\\
\textbf{Hidden: } \hspace{5pt} False\\
\textbf{Args: } \hspace{5pt} code/data/template


Shows link to source code and database.
\\ \\
User: Kaori, source code
\\
Kaori: Sure! Here is the link: https://github.com/vert178/kaoribot
\end{document}